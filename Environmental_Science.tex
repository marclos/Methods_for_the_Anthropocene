%----------------------------------------------------------
% EA Science Textbook and Laboratory Manual
% Version 0.2 Revised as a Lab Manual June 2016
% Version 0.3 Revised as Textbook Dec 2023

%----------------------------------------------------------

\documentclass{tufte-book}\usepackage[]{graphicx}\usepackage[]{xcolor}
% maxwidth is the original width if it is less than linewidth
% otherwise use linewidth (to make sure the graphics do not exceed the margin)
\makeatletter
\def\maxwidth{ %
  \ifdim\Gin@nat@width>\linewidth
    \linewidth
  \else
    \Gin@nat@width
  \fi
}
\makeatother

\definecolor{fgcolor}{rgb}{0.345, 0.345, 0.345}
\newcommand{\hlnum}[1]{\textcolor[rgb]{0.686,0.059,0.569}{#1}}%
\newcommand{\hlstr}[1]{\textcolor[rgb]{0.192,0.494,0.8}{#1}}%
\newcommand{\hlcom}[1]{\textcolor[rgb]{0.678,0.584,0.686}{\textit{#1}}}%
\newcommand{\hlopt}[1]{\textcolor[rgb]{0,0,0}{#1}}%
\newcommand{\hlstd}[1]{\textcolor[rgb]{0.345,0.345,0.345}{#1}}%
\newcommand{\hlkwa}[1]{\textcolor[rgb]{0.161,0.373,0.58}{\textbf{#1}}}%
\newcommand{\hlkwb}[1]{\textcolor[rgb]{0.69,0.353,0.396}{#1}}%
\newcommand{\hlkwc}[1]{\textcolor[rgb]{0.333,0.667,0.333}{#1}}%
\newcommand{\hlkwd}[1]{\textcolor[rgb]{0.737,0.353,0.396}{\textbf{#1}}}%
\let\hlipl\hlkwb

\usepackage{framed}
\makeatletter
\newenvironment{kframe}{%
 \def\at@end@of@kframe{}%
 \ifinner\ifhmode%
  \def\at@end@of@kframe{\end{minipage}}%
  \begin{minipage}{\columnwidth}%
 \fi\fi%
 \def\FrameCommand##1{\hskip\@totalleftmargin \hskip-\fboxsep
 \colorbox{shadecolor}{##1}\hskip-\fboxsep
     % There is no \\@totalrightmargin, so:
     \hskip-\linewidth \hskip-\@totalleftmargin \hskip\columnwidth}%
 \MakeFramed {\advance\hsize-\width
   \@totalleftmargin\z@ \linewidth\hsize
   \@setminipage}}%
 {\par\unskip\endMakeFramed%
 \at@end@of@kframe}
\makeatother

\definecolor{shadecolor}{rgb}{.97, .97, .97}
\definecolor{messagecolor}{rgb}{0, 0, 0}
\definecolor{warningcolor}{rgb}{1, 0, 1}
\definecolor{errorcolor}{rgb}{1, 0, 0}
\newenvironment{knitrout}{}{} % an empty environment to be redefined in TeX

\usepackage{alltt}

\usepackage{amsmath}
\usepackage{amsfonts}
\usepackage{amssymb}
\usepackage{graphicx}
%\usepackage[hypertex]{hyperref}

\usepackage{comment}
\usepackage{natbib}
\usepackage{lipsum}
\usepackage{units}
\usepackage{booktabs}

%%
% Prints a trailing space in a smart way.
\usepackage{xspace}
\usepackage{fancyvrb}
\fvset{fontsize=\normalsize}

\setkeys{Gin}{width=\linewidth,totalheight=\textheight,keepaspectratio}
\graphicspath{{graphics/}}

%----------------------------------------------------------

%-----------------------------------------------------------
%%
% If they're installed, use Bergamo and Chantilly from www.fontsite.com.
% They're clones of Bembo and Gill Sans, respectively.
\IfFileExists{bergamo.sty}{\usepackage[osf]{bergamo}}{}% Bembo
\IfFileExists{chantill.sty}{\usepackage{chantill}}{}% Gill Sans

%\usepackage{microtype}

%%
% Prints argument within hanging parentheses (i.e., parentheses that take
% up no horizontal space).  Useful in tabular environments.
\newcommand{\hangp}[1]{\makebox[0pt][r]{(}#1\makebox[0pt][l]{)}}

%%
% Prints an asterisk that takes up no horizontal space.
% Useful in tabular environments.
\newcommand{\hangstar}{\makebox[0pt][l]{*}}



%%
% Some shortcuts
\newcommand{\epa}{\textit{EPA}\xspace}

% Prints the month name (e.g., January) and the year (e.g., 2008)
\newcommand{\monthyear}{%
  \ifcase\month\or January\or February\or March\or April\or May\or June\or
  July\or August\or September\or October\or November\or
  December\fi\space\number\year
}


% Prints an epigraph and speaker in sans serif, all-caps type.
\newcommand{\openepigraph}[2]{%
  %\sffamily\fontsize{14}{16}\selectfont
  \begin{fullwidth}
  \sffamily\large
  \begin{doublespace}
  \noindent\allcaps{#1}\\% epigraph
  \noindent\allcaps{#2}% author
  \end{doublespace}
  \end{fullwidth}
}

% Inserts a blank page
\newcommand{\blankpage}{\newpage\hbox{}\thispagestyle{empty}\newpage}



% Typesets the font size, leading, and measure in the form of 10/12x26 pc.
\newcommand{\measure}[3]{#1/#2$\times$\unit[#3]{pc}}

% Macros for typesetting the documentation

\newcommand{\hlred}[1]{\textcolor{Maroon}{#1}}% prints in red
\newcommand{\hangleft}[1]{\makebox[0pt][r]{#1}}
\newcommand{\hairsp}{\hspace{1pt}}% hair space
\newcommand{\hquad}{\hskip0.5em\relax}% half quad space
\newcommand{\TODO}{\textcolor{red}{\bf TODO!}\xspace}
\newcommand{\na}{\quad--}% used in tables for N/A cells

%\begin{comment}
\providecommand{\XeLaTeX}{X\lower.5ex\hbox{\kern-0.15em\reflectbox{E}}\kern-0.1em\LaTeX}
\newcommand{\tXeLaTeX}{\XeLaTeX\index{XeLaTeX@\protect\XeLaTeX}}
% \index{\texttt{\textbackslash xyz}@\hangleft{\texttt{\textbackslash}}\texttt{xyz}}
\newcommand{\tuftebs}{\symbol{'134}}% a backslash in tt type in OT1/T1
\newcommand{\doccmdnoindex}[2][]{\texttt{\tuftebs#2}}% command name -- adds backslash automatically (and doesn't add cmd to the index)
\newcommand{\doccmddef}[2][]{%
  \hlred{\texttt{\tuftebs#2}}\label{cmd:#2}%
  \ifthenelse{\isempty{#1}}%
    {% add the command to the index
      \index{#2 command@\protect\hangleft{\texttt{\tuftebs}}\texttt{#2}}% command name
    }%
    {% add the command and package to the index
      \index{#2 command@\protect\hangleft{\texttt{\tuftebs}}\texttt{#2} (\texttt{#1} package)}% command name
      \index{#1 package@\texttt{#1} package}\index{packages!#1@\texttt{#1}}% package name
    }%
}% command name -- adds backslash automatically
\newcommand{\doccmd}[2][]{%
  \texttt{\tuftebs#2}%
  \ifthenelse{\isempty{#1}}%
    {% add the command to the index
      \index{#2 command@\protect\hangleft{\texttt{\tuftebs}}\texttt{#2}}% command name
    }%
    {% add the command and package to the index
      \index{#2 command@\protect\hangleft{\texttt{\tuftebs}}\texttt{#2} (\texttt{#1} package)}% command name
      \index{#1 package@\texttt{#1} package}\index{packages!#1@\texttt{#1}}% package name
    }%
}% command name -- adds backslash automatically
\newcommand{\docopt}[1]{\ensuremath{\langle}\textrm{\textit{#1}}\ensuremath{\rangle}}% optional command argument
\newcommand{\docarg}[1]{\textrm{\textit{#1}}}% (required) command argument
\newenvironment{docspec}{\begin{quotation}\ttfamily\parskip0pt\parindent0pt\ignorespaces}{\end{quotation}}% command specification environment
\newcommand{\docenv}[1]{\texttt{#1}\index{#1 environment@\texttt{#1} environment}\index{environments!#1@\texttt{#1}}}% environment name
\newcommand{\docenvdef}[1]{\hlred{\texttt{#1}}\label{env:#1}\index{#1 environment@\texttt{#1} environment}\index{environments!#1@\texttt{#1}}}% environment name
\newcommand{\docpkg}[1]{\texttt{#1}\index{#1 package@\texttt{#1} package}\index{packages!#1@\texttt{#1}}}% package name
\newcommand{\doccls}[1]{\texttt{#1}}% document class name
\newcommand{\docclsopt}[1]{\texttt{#1}\index{#1 class option@\texttt{#1} class option}\index{class options!#1@\texttt{#1}}}% document class option name
\newcommand{\docclsoptdef}[1]{\hlred{\texttt{#1}}\label{clsopt:#1}\index{#1 class option@\texttt{#1} class option}\index{class options!#1@\texttt{#1}}}% document class option name defined
\newcommand{\docmsg}[2]{\bigskip\begin{fullwidth}\noindent\ttfamily#1\end{fullwidth}\medskip\par\noindent#2}
\newcommand{\docfilehook}[2]{\texttt{#1}\index{file hooks!#2}\index{#1@\texttt{#1}}}
\newcommand{\doccounter}[1]{\texttt{#1}\index{#1 counter@\texttt{#1} counter}}
%\end{comment}
% Generates the index
\usepackage{makeidx}
\makeindex
\IfFileExists{upquote.sty}{\usepackage{upquote}}{}
\begin{document}

\frontmatter
%\title{Analyzing Environments}
%\subtitle{Textbook and Laboratory Manual}
%\author[Marc Los Huertos]{Marc Los Huertos, Branwen Williams, and Colin Robins}

\publisher{Honnold Library}

\title[EA Core]{Environmental Analysis Core EA10/EA20/EA30}
%\subtitle{Textbook and Laboratory Manual}
\author[Marc Los Huertos]{Professor Marc Los Huertos}
%\address[Pomona]{Environmental Analysis Program, Seely G Mudd%
%\hspace*{\fill}\linebreak\indent%
%640 N College Ave., Claremont, CA 91711}%
%\curraddr[M. Los Huertos]{Author One, current address, line 1
%\hspace*{\fill}\linebreak
%\indent Author One, current address, line 2}%
%\email[Marc Los Huertos]{marc.loshuertos@pomona.edu}
%\urladdr{http://www.ea.cuc.edu}

%\author{EA Faculty---Pitzer College}
%\address[Pitzer]{Environmental Analysis Program, %
%\hspace*{\fill}\linebreak\indent%
%XX, Claremont, CA 91711}%
\thanks{The Authors thank a million people!}
%\subjclass{Primary 05C38, 15A15; Secondary 05A15, 15A18}

% \author{EA Faculty---Keck Sciences}
% \address[Keck]{Environmental Analysis Program, %
% \hspace*{\fill}\linebreak\indent%
% 7th Street, Claremont, CA 91711}%
%\thanks{The Authors thank a million people!}
%\subjclass{Primary 05C38, 15A15; Secondary 05A15, 15A18}



\cleardoublepage
%\thispagestyle{empty}
\vspace*{13.5pc}
\begin{center}
%\dedicatory{Dedicated to Professor Richard Hazlett}
\end{center}
\cleardoublepage



\begin{comment}
% Front matter
%\frontmatter

% r.1 blank page
%\blankpage

% v.2 epigraphs
%\newpage\thispagestyle{empty}
%\openepigraph{Paul Rand}%, {\itshape Design, Form, and Chaos}}
%\vfill
%\openepigraph{Antoine de Saint-Exup\'{e}ry}
%\vfill
%\openepigraph{%
%\ldots the designer of a new system must not only be the implementor and the first
%large-scale user; the designer should also write the first user manual\ldots
If I had not participated fully in all these activities,
literally hundreds of improvements would never have been made,
because I would never have thought of them or perceived
why they were important.
%}{Donald E. Knuth}


\end{comment}

% r.3 full title page
\maketitle


% v.4 copyright page
\newpage
\begin{fullwidth}
~\vfill
\thispagestyle{empty}
\setlength{\parindent}{0pt}
\setlength{\parskip}{\baselineskip}
Copyright \copyright\ \the\year\ \thanklessauthor

\par\smallcaps{Published by \thanklesspublisher}

\par\smallcaps{tufte-latex.github.io/tufte-latex/}

\par Licensed under the Apache License, Version 2.0 (the ``License''); you may not
use this file except in compliance with the License. You may obtain a copy
of the License at \url{http://www.apache.org/licenses/LICENSE-2.0}. Unless
required by applicable law or agreed to in writing, software distributed
under the License is distributed on an \smallcaps{``AS IS'' BASIS, WITHOUT
WARRANTIES OR CONDITIONS OF ANY KIND}, either express or implied. See the
License for the specific language governing permissions and limitations
under the License.\index{license}

\par\textit{First printing, \monthyear}
\end{fullwidth}

% r.5 contents
\tableofcontents

\listoffigures

\listoftables

% r.7 dedication
\cleardoublepage
~\vfill
\begin{doublespace}
\noindent\fontsize{18}{22}\selectfont\itshape
\nohyphenation
Dedicated to those who appreciate \LaTeX{}
and the work of \mbox{Edward R.~Tufte}
and \mbox{Donald E.~Knuth}.
\end{doublespace}
\vfill
\vfill


% r.9 introduction
\cleardoublepage


\chapter*{Introduction}

This sample book discusses the design of Edward Tufte's
books\citep{Tufte2001,Tufte1990,Tufte1997,Tufte2006}
and the use of the \doccls{tufte-book} and \doccls{tufte-handout} document classes.


%%
% Start the main matter (normal chapters)

\include{Preface}

\mainmatter

\include{Anthropocene}

\part{Legacies in Environmental Science}

\include{Nuclear_Test_Ban}

\include{Clean_Air}

\include{Clean_Water}

\include{Safe_Drinking_Water}

\chapter{ESA and the Bald Eagle}

\chapter{Magnasun-Stevens and Fisheries}

\chapter{CERCLA and Love Canal}

\chapter{Montreal Protocol and CFCs}

%------------------------------------------------
\part{Awkward Progress in Environmental Sciences}

%\chapter{Progress and Western Science}

\include{chapters/Decolonizing_Environmental_Science}

\chapter{Acid Rain and SOx and NOx}

\begin{table}[htbp]
	\centering
	\caption{Learning Goals Matrix}
	\label{tab:LearningGoalsMatrix}
		\begin{tabular}{|l|p{4cm}|l|}\hline
Goals 					& Content/Activity				& Assessment \\ \hline\hline
Knowledge				&													&							\\
Comprehension		&													&							\\
Application			&	N Deposition						&							\\
Synthesis				&													&							\\
Evaluation			&													&							\\ \hline
		\end{tabular}

\end{table}

\chapter{Fate and Transport of Toxics}

\section{VOCs}

\section{Pesticides}

\section{Heavy Metals}

\section{Radionuclides}

Cesium-137 Semi Valley

\chapter{Clear Air and Progressive Science}

\section{The case for PM10, PM2.5, and PM0.1}

\section{Mercury}

\chapter{Ecology}

\section{Ecosystems Description}

\section{Ecosystem Function}

\section{Ecosystem Change}

\section{Degradation of Ecosystem Services and Loss of Biodiversity}

The Sixth Extinction: An Unnatural History


\chapter{Key resources}

\section{Water Supplies}

\subsection{Dams and Irrigation}

Dam Nation: How Water Shaped The West And Will Determine Its Future

\section{Soils}

Montgomery Is ag. eroding civilization?



\section{Atmosphere-Ocean Climate System}

\chapter{Pollution}

Toxic Truth: A Scientist, a Doctor, and the Battle over Lead

\section{Air pollution}

\subsection{Taxes, Cap \& Trade}



\section{Water Pollution}

Tom's River

\subsection{Surface Waters}

\subsection{Groundwater}




%------------------------------
\part{Regulatory Failures}

\chapter{Wilderness Act and Roads}

\chapter{Wetland Protection}

\chapter{Water Supply}

\section{Surface Supplies}

\section{Groundwater}

\section{Reverse Osmosis}

\chapter{Toxics: NIMBY and Social Justice}

\section{Toxic Siting}


\section{Vadose Zone Hydrology and Toxic Plume}


\chapter{Life Cycle Analysis, Recycling, and Waste}

\url{http://www.amazon.com/The-Works-Anatomy-Kate-Ascher/dp/0143112708/ref=cm_cr_pr_product_top?ie=UTF8}

\section{Solid Wastes}


\section{Nuclear Wastes}


\section{Economics of Planned Obsolescence}


\chapter{Cross Media Impacts of N}

\chapter{Agriculture and Food Production}

\section{Domestication}

\section{Green Revolution}

\subsection{Norman Borlaug and HYVs}

\subsection{Rachel Carlson and DDT}

\subsection{J.I. Rodale and Organic Agriculture}

\section{Promises of Biotechnology}


\chapter{Fisheries}

\section{Sustainable yield and its discontents}

\section{Marine Protected Areas}

\chapter{Human Population and Resource Use}

\section{Malthus Revisited}


\chapter{Wilderness Act and Roads}

\chapter{Wetland Protection}

\chapter{Water Supply}

\section{Surface Supplies}

\section{Groundwater}

\section{Reverse Osmosis}

\include{Toxics}

\chapter{Life Cycle Analysis, Recycling, and Waste}

\url{http://www.amazon.com/The-Works-Anatomy-Kate-Ascher/dp/0143112708/ref=cm_cr_pr_product_top?ie=UTF8}

\section{Solid Wastes}


\section{Nuclear Wastes}


\section{Economics of Planned Obsolescence}


\chapter{Cross Media Impacts of N}

\chapter{Agriculture and Food Production}

\section{Domestication}

\section{Green Revolution}

\subsection{Norman Borlaug and HYVs}

\subsection{Rachel Carlson and DDT}

\subsection{J.I. Rodale and Organic Agriculture}

\section{Promises of Biotechnology}


\chapter{Fisheries}

\section{Sustainable yield and its discontents}

\section{Marine Protected Areas}

\chapter{Human Population and Resource Use}

\section{Malthus Revisited}


\part{Climate Change: Environmental Science's Emerging Challenges}

\chapter{Energy: Supply and Impacts}

\section{Demand and Technology Changes}

\section{Intended and Unintended Xonsequences}

\chapter{Climate and Temperature Records}


\chapter{Anthropogenic Radiative Forcing: Carbon Dioxide, Methane, Nitrous Oxide}

\chapter{Climate Change Impacts: Water}

\chapter{Climate Change Impacts: Biodiversity}


\chapter{Energy: Supply and Impacts}

\section{Demand and Technology Changes}

\section{Intended and Unintended Consequences}

\chapter{Climate and Temperature Records}


\chapter{Anthropogenic Radiative Forcing: Carbon Dioxide, Methane, Nitrous Oxide}

\chapter{Climate Change Impacts: Water}

\chapter{Climate Change Impacts: Biodiversity}




\chapter*{Preface}

\markboth{PREFACE}{PREFACE} I (marc) have begun trying to document our Pomona/5C EA curriculum in an attempt to document the readings, activities, and goals as we try to prepare students to address the environmental issues that face the region, nation and globe. As a priority, this begins as a way to develop a laboratory manual -- for an interdisciplinary program.

\section{Pedagogical Framework}

I root this approach in a decidedly ``structured pedagogical'' framework to help me align student learning with assessment and education theory. It's my own bias and is not meant to constrain the conversations about this topic.


\subsection{Features of Threshold Concepts}

Transformative: Once understood, a threshold concept changes the way in which the student views the discipline.  A threshold concept is fundamentally transformative: it may be considered ‘akin to a portal, opening up a new and previously inaccessible way of thinking about something . . . it represents a transformed way of understanding, or interpreting, or viewing . . . without which the learner cannot progress’ (Meyer and Land, 2003).

On mastering a threshold concept the learner begins to think as does a professional in that discipline and not simply as a student of that discipline - the learner begins to think like an engineer... like a chemist ... like an economist ..., i.e. grasping a threshold concept involves both an ontological as well as a conceptual shift.

Troublesome: Threshold concepts are likely to be troublesome for the student. Perkins [1999, 2006] has suggested that knowledge can be troublesome e.g. when it is counter-intuitive, alien or seemingly incoherent. Mastering threshold concepts often requires the acquisition of knowledge that is troublesome. Depending on discipline and context, this knowledge might be counter-intuitive, alien, tacit, ritualised, inert, conceptually difficult, characterized by an inaccessible ‘underlying game’, characterised by supercomplexity or perhaps troublesome because the learner remains ‘defended’ and does not wish to change or let go of their customary way of seeing things (Perkins 2006, Land, Cousin, Meyer \& Davies 2005, Land, Meyer \& Baillie 2010).

The following instances of troublesome knowledge have been taken from examples of troublesome knowledge revealed in threshold concept studies in engineering and science.

Inert, ritual or alien knowledge

Cheek has suggested that geological time (deep time) represents an example of knowledge that might be inert (retrievable for an examination test but not not connected to other ideas or transferable to real world experiences), and/or ritual (routinely applied but with little or no meaning) and/or alien (emanating from another culture or discourse).

Deep time might be considered ritual and/or inert knowledge, particularly if students have memorized an age for the earth but then proceed to reason about deep time as if that age was irrelevant. More likely deep time can be considered both conceptually difficult and alien. A human lifespan is inconsequential in terms of geologic process such as mountain building or the sculpting of the Grand Canyon. To grasp that rocks behave plastically, continents move, and the mountains we visit will one day be gone are all conceptually difficult. Cheek (2010)

Alien knowledge

Carbon atom	For many engineering and science students (and some staff) quantum mechanics is a rich source of alien knowledge, i.e. knowledge emanating from another culture or discourse, in this case quantum mechanics versus classical mechanics where the latter appears to gel, most of the time, with a 'common sense' expectation and where the former certainly does not. Park and Light (2009) have shown that probability when discussing electronic orbitals and energy quantization is very troublesome, in this manner, for many of their first year chemistry students.

Dyce, Pegwell Bay

The ‘defended’ student
In the UK many engineering students meet only a definition of a linear one-dimensional electric field, −ΔV/Δx, in their school physics courses. They rapidly need to be able to handle non-linear electric fields, e.g. −∂V/∂r equated to a non-linear function of r or even −∇V, in their first year at University. In the experience of this author (Flanagan), a significant number continue to attempt to solve problems using the ‘school’ linear equation, when a more advanced non-linear equation is required, for quite some time after they have been taught about non-linear fields. This reluctance to let go of their ‘school physics’ view hinders them, not only in exercises on electric fields, but, of course, in grasping many aspects of electronic and electrical engineering, e.g. the functioning of electronic devices such as the transistor. Flanagan and Smith (to be published) have examined this reluctance in terms of Bruner’s spiral curriculum and Flanagan et al. (2010) have illustrated the importance of considering such preliminal variation in identifying the compounded threshold concept associated with the mastery of transmission lines.

      V Lombardi quote
Transmission lines      	Counter-intuitive knowledge
Many electrical engineering students find the concept of reactive power counter-intuitive. A power that is mathematically imaginary and appears to be an oxymoron (a wattless power that can do no work appears to be a 'powerless power’) is troublesome, yet without a grasp of its meaning, role and significance a student will never truly understand power circuits, hence frustratingly troublesome. Flanagan, Taylor and Meyer (2010) have examined the role of reactive power in identifying the compounded threshold concept associated with the mastery of transmission lines.


Irreversible: Given their transformative potential, threshold concepts are also likely to be irreversible, i.e. they are difficult to unlearn. Given their transformative potential, threshold concepts are also likely to be irreversible, i.e. they are difficult to unlearn.

The Academy of Art University, San Francisco, present an excellent one page discussion of Threshold Concepts in which they offer the following simple analogy:
Did you know that there is an arrow embedded in the FedEx logo? Maybe you did. If you didn’t, look for it next time. Once you see it, you will never be able to look at the FedEx logo again without the arrow popping out at you. You will not understand how you ever missed it.
Look for the arrow in the FedEx logo:



Click here to return to the original logo.



BUT, as the Academy of Art University authors themselves stress, this is an amusing but overly simple analogy as a student's mastery of a threshold concept is unlikely to involve instantaneous recognition:
We would argue for the notion of learning as excursive, as a journey or excursion which will have intended direction and outcome but will also acknowledge (and indeed desire) that there will be deviation and unexpected outcomes within the excursion; there will be digression and revisiting (recursion) and possible further points of departure and revised direction. The eventual destination may be reached, or it may be revised. It may be a surprise. It will certainly be the point of embarkation for further excursion.
                                    Ray Land, Glynis Cousin, Erik Meyer \& Peter Davies
                                    Threshold concepts and troublesome knowledge (3): implications for course design and evaluation
In short, there is no simple passage in learning from ‘easy’ to ‘difficult’; mastery of a threshold concept often involves messy journeys back, forth and across conceptual terrain
                                    Glynis Cousin, An introduction to threshold concepts

The process of understanding threshold concepts and adjusting one's worldview is often very difficult. Unlike the instant recognition of the arrow in the FedEx logo, understanding a threshold concept is more of a journey of questioning (and perhaps self-doubt), during which students typically go back and forth between using their old lens ("common sense") and their new lens (the threshold concept) until the new concept is comfortable and familiar enough to rely on. Teachers need to be especially supportive of students as they progress through the understanding of a threshold concept. Then and only then can students progress in their understanding of a discipline, and ultimately become contributing members in a field.
                                    The Academy of Art University: Identifying Threshold Concepts

The suggestion that threshold concepts are irreversible is supported by the difficulty that some lecturers find in recognizing threshold concepts that students may encounter in their courses; to quote Cousin again,
One of the difficulties teachers have is that of retracing the journey back to their own days of ‘innocence’, when understandings of threshold concepts escaped them in the early stages of their own learning.
                                    Glynis Cousin, An introduction to threshold concepts

Integrative: Threshold concepts, once learned, are likely to bring together different aspects of the subject that previously did not appear, to the student, to be related. Threshold concepts, once learned, are likely to bring together different aspects of the subject that previously did not appear, to the student, to be related.

In engineering the integrative aspect may be seen at two levels.

1.	A threshold concept may bring together several troublesome concepts:
Phasors are certainly integrative − relying on connections between complex numbers, sinusoidal properties and exponentials. The application of complex numbers to real components − inductors and capacitors − is initially alien knowledge to students (Meyer and Land, 2003).

Holloway, Alpay \& Bull (2010)

The confluence of several examples of troublesome knowledge within a threshold concept has been observed by several investigators in researching the teaching of electrical engineering.

Flanagan, Taylor \& Meyer (2010), have identified a threshold concept in the mastery of transmission line theory which is characterised by:

Troublesome visualisations and abstractions

Troublesome oscillation between the abstract and the concrete

Complex arithmetic

Counter-intuitive entities − reactive power, characteristic impedance
They refer to such a concept as compounded threshold concept

Bernhard \& Carstensen (2007), in extensive studies of the teaching of circuit analysis, have developed the idea of complex concepts.

2.	There are several concepts, that appear in two or more modules throughout a typical engineering syllabus, which if grasped help integrate the discipline. Some of these, e.g. Laplace transform in circuit analysis, circuit design, control systems, signal processing; phasors in circuit analysis, circuit design, communication systems, power electronics, are troublesome for many students and may be threshold concepts or contribute to a compounded threshold concept.

The integrative element of such threshold concepts may be examined by observing the facilitation of a student’s grasp of the concept in one module after completing an earlier potentially integrative module.

Bounded: A threshold concept will probably delineate a particular conceptual space, serving a specific and limited purpose. A threshold concept will probably delineate a particular conceptual space, serving a specific and limited purpose (Jan Smith 2006).

Our discussions with practitioners in a range of disciplinary areas have led us to conclude that a threshold concept, across a range of subject contexts, is likely to be:

d) Possibly often (though not necessarily always) bounded in that any conceptual space will have terminal frontiers, bordering with thresholds into new conceptual areas. Siân Bayne suggests that such boundedness may in certain instances serve to constitute the demarcation between disciplinary areas, to define academic territories.
                                                                    Erik Meyer and Ray Land (2005)
guedalla quote

Boundedness may be associated with a discipline’s special language:

The notion of boundedness may best be illustrated by the use of specialist terminology that acquires a meaning in one subject that clashes with everyday usage. One such as example, relating to computing, is the term ‘deprecate’. Whilst common usage imbues this word with negative connotations, in computing it simply means to let an aspect of programming gently wither away, e.g. the retention of an outdated method along with its replacements for many revisions and updates of a programming language.
                                                                    Mick Flanagan and Jan Smith (2008)

Discursive: Meyer and Land [10] suggest that the crossing of a threshold will incorporate an enhanced and extended use of language. Meyer and Land suggest that the crossing of a threshold will incorporate an enhanced and extended use of language.

It is hard to imagine any shift in perspective that is not simultaneously accompanied by (or occasioned through) an extension of the student's use of language. Through this elaboration of discourse new thinking is brought into being, expressed, reflected upon and communicated. This extension of language might be acquired, for example, from that in use within a specific discipline, language community or community of practice, or it might, of course, be self-generated. It might involve natural language, formal language or symbolic language.
Erik Meyer and Ray Land (2005)


In the engineering and the science disciplines this enhanced use of language is often reflected in the student’s acquisition and meaningful use of a professional vocabulary, i.e. free from mimicry. This may be observed in their formal presentations and their response to the subsequent questions and in their changed lanquage in conversations with fellow students:

We have repeatedly seen students who have grasped a local threshold concept themselves enthusiastically and volubly attempt to lift their partners over the same threshold
                              Mick Flanagan and Jan Smith (2008)
                              Writing about engineering students learning to program

Reconstitutive: ``Understanding a threshold concept may entail a shift in learner subjectivity, which is implied through the transformative and discursive aspects already noted. Such reconstitution is, perhaps, more likely to be recognised initially by others, and also to take place over time (Smith)''.  Understanding a threshold concept may entail a shift in learner subjectivity, which is implied through the transformative and discursive aspects already noted. Such reconstitution is, perhaps, more likely to be recognised initially by others, and also to take place over time.
                                                                                              Jan Smith (2006)

Within the liminal state an integration of new knowledge occurs which requires a reconfiguring of the learner’s prior conceptual schema and a letting go or discarding of any earlier conceptual stance. This reconfiguration occasions an ontological and an epistemic shift. The integration/reconfiguration and accompanying ontological/epistemic shift can be seen as reconstitutive features of the threshold concept:

 liminality diagram



We would not, however, wish to imply that this relational view has an overly rigid sequential nature. It has been emphasised elsewhere (Land et al, 2005) that the acquisition of threshold concepts often involves a degree of recursiveness, and of oscillation, which would need to be layered across this simple diagram [move the cursor across the above figure].
                                                    Ray Land, Erik Meyer and Caroline Baillie (2010)
In short, there is no simple passage in learning from ‘easy’ to ‘difficult’; mastery of a threshold concept often involves messy journeys back, forth and across conceptual terrain
                                                                                              Glynis Cousin (2006)

Liminality: Meyer and Land [13] have likened the crossing of the pedagogic threshold to a ‘rite of passage’ (drawing on the ethnographical studies of Gennep and Turner in which a transitional or liminal space has to be traversed; “in short, there is no simple passage in learning from ‘easy’ to ‘difficult’; mastery of a threshold concept often involves messy journeys back, forth and across conceptual terrain. (Cousin [7])”. Difficulty in understanding threshold concepts may leave the learner in a state of ‘liminality’, a suspended state of partial understanding, or ‘stuck place’, in which understanding approximates to a kind of ‘mimicry’ or lack of authenticity. Insights gained by learners as they cross thresholds can be exhilarating but might also be unsettling, requiring an uncomfortable shift in identity, or, paradoxically, a sense of loss. A further complication might be the operation of an ‘underlying game’ which requires the learner to comprehend the often tacit games of enquiry or ways of thinking and practising inherent within specific disciplinary
                                                                                Land, Meyer and Baillie (2010)

Liminal states: This space is likened to that which adolescents inhabit: - not yet adults; not quite children. It is an unstable space in which the learner may oscillate between old and emergent understandings just as adolescents often move between adult-like and child-like responses to their transitional status. But once a learner enters this liminal space, she is engaged with the project of mastery unlike the learner who remains in a state of pre-liminality in which understandings are at best vague. The idea that learners enter into a liminal state in their attempts to grasp certain concepts in their subjects presents a powerful way of remembering that learning is both affective and cognitive and that it involves identity shifts which can entail troublesome, unsafe journeys. Often students construct their own conditions of safety through the practice of mimicry. In our research, we came across teachers who lamented this tendency among students to substitute mimicry for mastery (Cousin, 2006b, p.139).
                                                                                              Cousin (2006a)

If viewed as a journey through preliminal, liminal and postliminal states, the features that characterise threshold concepts can now be represented relationally:

 liminality diagram



We would not, however, wish to imply that this relational view has an overly rigid sequential nature. It has been emphasised elsewhere (Land et al, 2005) that the acquisition of threshold concepts often involves a degree of recursiveness, and of oscillation, which would need to be layered across this simple diagram [move the cursor across the above figure].
                                                                                              Land, Meyer and Baillie (2010)
In short, there is no simple passage in learning from ‘easy’ to ‘difficult’; mastery of a threshold concept often involves messy journeys back, forth and across conceptual terrain
                                                                                              Cousin (2006a)



Threshold Concepts:


Evidence of Success:

Word Bank of Success:



\subsection{Project-based Learning}

Problem-based learning (PBL) usually has the several specific attributes such as being student-centered, taking place in small groups with the faculty acting as a facilitator, and being organized around problems.

\begin{enumerate}
	\item Authentic and Realistic
	\item Constructive and Integrated
	\item Suitable Complexity
	\item Promoting Self-directed learning and lifelong learning
	\item Stimulate Critical Thinking and Metacognitive Skills
\end{enumerate}

Another list includes

\begin{enumerate}
	\item Challenging and Complex Problem or Question
	\item Authenticity
	\item Student Voice and Choice
	\item Critique and Revision
	\item Public Product
\end{enumerate}


The curriculum is structured in thematic blocks, in which the semester is divided into a series of periods of approximately six weeks, and each period focuses on a particular theme. A series of cases are planned for the students to work on in each period. The students themselves choose to analyse one of the cases, which in turn can be done both orally and in writing. The subject disciplines are integrated through relating the case to professional practice. For example, in the field of medicine, the starting point is often a description of the patient. In Maastricht, the `Seven Step' method was developed to help students analyse the problem:

1. clarify the concepts;
2. define the problem;
3. analyse the problem;
4. find the explanation;
5. formulate the learning objective;
6. search for further information; and
7. report and test new information.

Outcome-Based Science, Technology, Engineering, and Mathematics Education


 Characteristics of Problem-Based
Learning*
ERIK DE GRAAFF
Delft University of Technology, the Netherlands
ANETTE KOLMOS
Aalborg University, Denmark


Wiek, Arnim, et al. "Integrating problem-and project-based learning into sustainability programs: A case study on the School of Sustainability at Arizona State University." International Journal of Sustainability in Higher Education 15.4 (2014): 431-449.

Traditional categories of Bloom's taxonomy...
\begin{enumerate}
	\item Knowledge
	\item Comprehension
	\item Application
	\item Analysis
	\item Synthesis
	\item Evaluation
\end{enumerate}

In particular the verbs, or actions embedded for each of Bloom's taxonomy are used to develop learning outcomes as highlighted in Figure~\ref{fig:Bloom}.

\begin{figure}
	\centering
		\includegraphics[width=1.00\textwidth]{Figures/640px-Blooms_rose_svg.eps}
	\caption{Bloom's Taxonomy}
	\label{fig:Bloom}
\end{figure}

The chapters are organized to promote the following learning goals that include deeper learning with application, evaluation, and creation (Figure~\ref{fig:CognitiveDomains}).

\begin{figure}
	\centering
		\includegraphics[width=1.00\textwidth]{Figures/653px-BloomsCognitiveDomain_svg.eps}
	\caption{Cogniative Domains}
	\label{fig:CognitiveDomains}
\end{figure}


\section{Reasoning in Science}

Inductive reasoning (as opposed to deductive reasoning) is reasoning in which the premises seek to supply strong evidence for (not absolute proof of) the truth of the conclusion. While the conclusion of a deductive argument is supposed to be certain, the truth of the conclusion of an inductive argument is supposed to be probable, based upon the evidence given.


Deductive reasoning links premises with conclusions. If all premises are true, the terms are clear, and the rules of deductive logic are followed, then the conclusion reached is necessarily true. Deductive reasoning (top-down logic) contrasts with inductive reasoning (bottom-up logic) in the following way: In deductive reasoning, a conclusion is reached reductively by applying general rules that hold over the entirety of a closed domain of discourse, narrowing the range under consideration until only the conclusion is left. In inductive reasoning, the conclusion is reached by generalizing or extrapolating from initial information.
In addition, the goals of the courses provide learning outcomes that include the following categories.

\subsection{Required Skills to Promote EA Reasoning}

To be successful in addressing environmental issues, EA course should promote the following skills and abilities.

\begin{description}
	\item [Empirical Skills] a way of gaining knowledge by means of direct and indirect observation or experience. Empirical evidence (the record of one's direct observations or experiences) can be analyzed quantitatively or qualitatively.

	\item [Qualitative Methods] gather an in-depth understanding of behavior and the reasons that govern such behavior. The qualitative method investigates the why and how of decision making, not just what, where, when. Hence, smaller but focused samples are more often used than large samples. In the conventional view, qualitative methods produce information only on the particular cases studied, and any more general conclusions are only propositions (informed assertions). Quantitative methods can then be used to seek empirical support for such research hypotheses.

	\item [Quantitative Methods] relies on a systematic empirical investigation of social or natural phenomena via statistical, mathematical or numerical data or computational techniques. The objective of quantitative research is to develop and employ mathematical models, theories and/or hypotheses pertaining to phenomena. The process of measurement is central to quantitative research because it provides a connection between empirical observation and mathematical expression of quantitative relationships that might be used for prediction.

	\item [Analytical Skills] are the ability to visualize, articulate, and solve both complex and uncomplicated problems and concepts and make decisions that are sensible and based on available information. Such skills include demonstration of the ability to apply logical thinking to gathering and analyzing information, designing and testing solutions to problems, and formulating plans.

	\item [Critical Thinking] is an intellectually disciplined process of actively and skillfully conceptualizing, applying, analyzing, synthesizing, and/or evaluating information gathered from, or generated by, observation, experience, reflection, reasoning, or communication, as a guide to belief and action.

	\item [Communications Skills] is the activity of conveying information through the exchange of ideas, feelings, intentions, attitudes, expectations, perceptions or commands, as by speech, non-verbal gestures, writings, behavior and possibly by other means such as electromagnetic, chemical or physical phenomena. It is the meaningful exchange of information between two or more participants (machines, organisms or their parts).

\end{description}

The content for each of these will be outlined within a matrix at the beginning of each chapter.


\section{Learning Outcomes}

To evaluate our teaching and student learning, the EA program has developed a set of learning outcomes for the majors. As part of this exercise and in an effort to promote some ``consistency'' between campus offerings, we define the following outcomes...

\subsection{EA Student Learning Outcomes (Pomona College Example)}

A student who majors in Environmental Analysis will:

\begin{itemize}
	\item Engage, assess, and critique an interdisciplinary scholarly literature;
	\item apply relevant theoretical techniques and methodological insights to environmental issues across the disciplines;
	\item conduct original archival, empirical and/or applied research, individually and collaboratively;
	\item speak and write clearly and persuasively; and
	\item understand the real-world dimensions of environmental problem-solving.
\end{itemize}

\subsection{Course Student Learning Outcomes (Marc's EA30L Example)}

We will fulfill a subset of the Student Learning Outcomes with the following course learning outcomes (CLOs):
\begin{enumerate}
	\item Describe how human activities impact soil, water, air, climate, biota, and other resources;
	\item Evaluate and design effective environmental monitoring programs.
	\item Develop basic analytical skills to collect and analyze quantitative data.
	\item Interpret and communicate scientific information accurately and appropriately.
	\item Apply ethical norms that guide scientific practice.
\end{enumerate}

\mainmatter
%-------------------------------------------------------------



\chapter{Ecology}

\section{Ecosystems Description}

\section{Ecosystem Function}

\section{Ecosystem Change}

\section{Degradation of Ecosystem Services and Loss of Biodiversity}

The Sixth Extinction: An Unnatural History


\chapter{Key resources}

\section{Water Supplies}

\subsection{Dams and Irrigation}

Dam Nation: How Water Shaped The West And Will Determine Its Future

\section{Soils}

Montgomery Is ag. eroding civilization?



\section{Atmosphere-Ocean Climate System}

\chapter{Pollution}

Toxic Truth: A Scientist, a Doctor, and the Battle over Lead

\section{Air pollution}

\subsection{Taxes, Cap \& Trade}



\section{Water Pollution}

Tom's River

\subsection{Surface Waters}

\subsection{Groundwater}







%------------------------------------------------


\backmatter \appendix

\chapter{Quality Control and Quality Assurance}

\chapter{Laboratory Basics--Lab Safety}

\chapter{Laboratory Basics--Proper Use of Equipment and Instruments}

\chapter{Laboratory Basics--Accuracy and Precision}

\chapter{Field Sampling}

\section{Planning}

\section{Mapping and Land Use History}

\section{Sampling Procedures}

\subsection{Air Sampling}

\subsection{Water Sampling}

\subsection{Soil Sampling}


\subsection{Vadose Zone and Ground Water Sampling}

\chapter{Project 1: Waste Streams and Laboratory Safety}

\chapter{Project X: Soil Lead and Spatial Sampling}

\section{Methods}

\subsection{Team Assignments}

Each student has been assigned to a team below. Each team will assign roles as described below.

Before the field work you should assign roles for each member of your team:


Creating Random Numbers
Be sure to create your random numbers before Tuesday
Handout to generate random sampling points (Needs knitting)
GPS manager
Please test app on a smart phone before Tuesday
Requires "Location Services" to be switched on.
Soil Sampling Manager (Demo sampling next Tuesday)
If you can meet Marc at 12:45 PM that would be ideal
Sample Curator (with a plan submitted before Tuesday)
What are the sample ID?
How will data be recorded?
What will samples be preserved and stored?
Class Coordinator (Sara-ling) and POM-KS Coordinator (Thomas)
For each of these roles, you will submit a written summary describing what the role entails for each category to Sakai before Tuesday 1 pm, under Project 3: Part II. Field Prep. In addition, provide a detailed methodology so that your group will know exactly what needs to be done once you arrive in class, e.g. what are your random numbers, how will you sample the soil, what are the ID numbers, what do the COC forms look like and who will fill them out?

The following google pages have been developed for this project by Profs. Williams (KS) and Los Huertos (PO) to facilitate this project. These are draft versions and we may need to augment and improve them over the next few sessions. And if you find areas that are unclear, please let me know and I will try to fix any issues. These pages should be scanned by all of you, but they are also a resource for each of the team member assignments.  You may use these as templates for your work in developing the methods for the project -- but receive high marks, you should look at the peer review literature to ensure your methods are "vetted" and "time-honored".

\chapter{Project X: Ozone and Temporal Variation}

\chapter{Project X: Water Supply and Quality: Salinity}

\chapter{Project X: Vadose Zone and Groundwater Project}

Groundwater is a key resource throughout the world, however, we have failed to protect is with any rigor.

\chapter{Project X: Greenhouse Gases and Fertility}


\chapter{Basics of Chemistry}

\section{Periodic Table -- Rows and Columns}

\section{Units and Conversions}

\section{Mathematics and Text}

It holds \cite{KarelRektorys} the following
\begin{theorem}
(The Currant minimax principle.) Let $T$ be completely continuous
selfadjoint operator in a Hilbert space $H$. Let $n$ be an
arbitrary integer and let $u_1,\ldots,u_{n-1}$ be an arbitrary
system of $n-1$ linearly independent elements of $H$. Denote
\begin{equation}
\max_{\substack{v\in H, v\neq
0\\(v,u_1)=0,\ldots,(v,u_n)=0}}\frac{(Tv,v)}{(v,v)}=m(u_1,\ldots,
u_{n-1}) \label{eqn10}
\end{equation}
Then the $n$-th eigenvalue of $T$ is equal to the minimum of these
maxima, when minimizing over all linearly independent systems
$u_1,\ldots u_{n-1}$ in $H$,
\begin{equation}
\mu_n = \min_{\substack{u_1,\ldots, u_{n-1}\in H}} m(u_1,\ldots,
u_{n-1}) \label{eqn20}
\end{equation}
\end{theorem}

The above equations are automatically numbered as equation
(\ref{eqn10}) and (\ref{eqn20}).

\section{Reactions -- Directions and Rates}

\section{Notes on Radionuclides and Stable Isotopes}


\chapter[Experiments and Statistics]{Introduction to Experimental Design and Statistics}

\section{Sources of Variation}

\section{Descriptions of Central Tendencies}

\section{Spread of Data}

\section{Probabilities and Distributions}

\section{Confidence Intervals}

\section{Comparing two populations}

\section{Regression and best fit lines}

\section{Time Series}

\subsection{Smoothing}

\subsection{Decomposition}


\section{ANOVA}



This text is a sample for a short bibliography. You can cite a
book by making use of the command \verb"\cite{KarelRektorys}":
\cite{KarelRektorys}. Papers can be cited similarly:
\cite{Bertoti97}. If you want multiple citations to appear in a
single set of square brackets you must type all of the citation
keys inside a single citation, separating each with a comma. Here
is an example: \cite{Bertoti97, Szeidl2001, Carlson67}.

\begin{thebibliography}{9}
\bibitem {KarelRektorys}Rektorys, K., \textit{Variational methods in Mathematics,
Science and Engineering}, D. Reidel Publishing Company,
Dordrecht-Hollanf/Boston-U.S.A., 2th edition, 1975

\bibitem {Bertoti97} \textsc{Bert\'{o}ti, E.}:\ \textit{On mixed variational formulation
of linear elasticity using nonsymmetric stresses and
displacements}, International Journal for Numerical Methods in
Engineering., \textbf{42}, (1997), 561-578.

\bibitem {Szeidl2001} \textsc{Szeidl, G.}:\ \textit{Boundary integral equations for
plane problems in terms of stress functions of order one}, Journal
of Computational and Applied Mechanics, \textbf{2}(2), (2001),
237-261.

\bibitem {Carlson67}  \textsc{Carlson D. E.}:\ \textit{On G\"{u}nther's stress functions
for couple stresses}, Quart. Appl. Math., \textbf{25}, (1967),
139-146.
\end{thebibliography}

\end{document}


\backmatter

\appendix

\chapter{Laboratory Basics--Lab Safety}


\chapter{Quality Control and Quality Assurance}

\chapter{Laboratory Basics--Accuracy and Precision}

\chapter{Laboratory Basics--Proper Use of Equipment and Instruments}

\chapter{Field Sampling}

\section{Planning}

\section{Mapping and Land Use History}

\section{Sampling Procedures}

\subsection{Air Sampling}

\subsection{Water Sampling}

\subsection{Soil Sampling}


\subsection{Vadose Zone and Ground Water Sampling}

\chapter{Project 1: Waste Streams and Laboratory Safety}

\chapter{Project X: Soil Lead and Spatial Sampling}

\section{Methods}

\subsection{Team Assignments}

Each student has been assigned to a team below. Each team will assign roles as described below.

Before the field work you should assign roles for each member of your team:


Creating Random Numbers
Be sure to create your random numbers before Tuesday
Handout to generate random sampling points (Needs knitting)
GPS manager
Please test app on a smart phone before Tuesday
Requires "Location Services" to be switched on.
Soil Sampling Manager (Demo sampling next Tuesday)
If you can meet Marc at 12:45 PM that would be ideal
Sample Curator (with a plan submitted before Tuesday)
What are the sample ID?
How will data be recorded?
What will samples be preserved and stored?
Class Coordinator (Sara-ling) and POM-KS Coordinator (Thomas)
For each of these roles, you will submit a written summary describing what the role entails for each category to Sakai before Tuesday 1 pm, under Project 3: Part II. Field Prep. In addition, provide a detailed methodology so that your group will know exactly what needs to be done once you arrive in class, e.g. what are your random numbers, how will you sample the soil, what are the ID numbers, what do the COC forms look like and who will fill them out?

The following google pages have been developed for this project by Profs. Williams (KS) and Los Huertos (PO) to facilitate this project. These are draft versions and we may need to augment and improve them over the next few sessions. And if you find areas that are unclear, please let me know and I will try to fix any issues. These pages should be scanned by all of you, but they are also a resource for each of the team member assignments.  You may use these as templates for your work in developing the methods for the project -- but receive high marks, you should look at the peer review literature to ensure your methods are "vetted" and "time-honored".

\chapter{Project X: Ozone and Temporal Variation}

\chapter{Project X: Water Supply and Quality: Salinity}

\chapter{Project X: Vadose Zone and Groundwater Project}

Groundwater is a key resource throughout the world, however, we have failed to protect is with any rigor.

\chapter{Project X: Greenhouse Gases and Fertility}


\chapter{Basics of Chemistry}

\section{Periodic Table -- Rows and Columns}

\section{Units and Conversions}


\section{Reactions -- Directions and Rates}

\section{Notes on Radionuclides and Stable Isotopes}


\chapter[Experiments and Statistics]{Introduction to Experimental Design and Statistics}

\section{Sources of Variation}

\section{Descriptions of Central Tendencies}

\section{Spread of Data}

\section{Probabilities and Distributions}

\section{Confidence Intervals}

\section{Comparing two populations}

\section{Regression and best fit lines}

\section{Time Series}

\subsection{Smoothing}

\subsection{Decomposition}


\section{ANOVA}

\chapter{References}

\bibentry{LosHuertos2020ecology}

\bibliographystyle{cbe}
\bibliography{../../References}


\end{document}

