\documentclass{tufte-handout}\usepackage[]{graphicx}\usepackage[]{xcolor}
% maxwidth is the original width if it is less than linewidth
% otherwise use linewidth (to make sure the graphics do not exceed the margin)
\makeatletter
\def\maxwidth{ %
  \ifdim\Gin@nat@width>\linewidth
    \linewidth
  \else
    \Gin@nat@width
  \fi
}
\makeatother

\definecolor{fgcolor}{rgb}{0.345, 0.345, 0.345}
\newcommand{\hlnum}[1]{\textcolor[rgb]{0.686,0.059,0.569}{#1}}%
\newcommand{\hlstr}[1]{\textcolor[rgb]{0.192,0.494,0.8}{#1}}%
\newcommand{\hlcom}[1]{\textcolor[rgb]{0.678,0.584,0.686}{\textit{#1}}}%
\newcommand{\hlopt}[1]{\textcolor[rgb]{0,0,0}{#1}}%
\newcommand{\hlstd}[1]{\textcolor[rgb]{0.345,0.345,0.345}{#1}}%
\newcommand{\hlkwa}[1]{\textcolor[rgb]{0.161,0.373,0.58}{\textbf{#1}}}%
\newcommand{\hlkwb}[1]{\textcolor[rgb]{0.69,0.353,0.396}{#1}}%
\newcommand{\hlkwc}[1]{\textcolor[rgb]{0.333,0.667,0.333}{#1}}%
\newcommand{\hlkwd}[1]{\textcolor[rgb]{0.737,0.353,0.396}{\textbf{#1}}}%
\let\hlipl\hlkwb

\usepackage{framed}
\makeatletter
\newenvironment{kframe}{%
 \def\at@end@of@kframe{}%
 \ifinner\ifhmode%
  \def\at@end@of@kframe{\end{minipage}}%
  \begin{minipage}{\columnwidth}%
 \fi\fi%
 \def\FrameCommand##1{\hskip\@totalleftmargin \hskip-\fboxsep
 \colorbox{shadecolor}{##1}\hskip-\fboxsep
     % There is no \\@totalrightmargin, so:
     \hskip-\linewidth \hskip-\@totalleftmargin \hskip\columnwidth}%
 \MakeFramed {\advance\hsize-\width
   \@totalleftmargin\z@ \linewidth\hsize
   \@setminipage}}%
 {\par\unskip\endMakeFramed%
 \at@end@of@kframe}
\makeatother

\definecolor{shadecolor}{rgb}{.97, .97, .97}
\definecolor{messagecolor}{rgb}{0, 0, 0}
\definecolor{warningcolor}{rgb}{1, 0, 1}
\definecolor{errorcolor}{rgb}{1, 0, 0}
\newenvironment{knitrout}{}{} % an empty environment to be redefined in TeX

\usepackage{alltt}

\title{Earth System Science}
% \date{}
\IfFileExists{upquote.sty}{\usepackage{upquote}}{}
\begin{document}

\maketitle

\section{What is Earth System Science?}

Earth system science is an interdisciplinary field that emerged in the late 20th century, aiming to understand the interactions and processes that occur within the Earth's various subsystems. This scientific approach considers the Earth as a complex, dynamic system with interconnected components, including the atmosphere, hydrosphere, lithosphere, biosphere, and anthroposphere. Here's a brief overview of the history of Earth system science:

\subsection{Early Foundations (19th Century)}

Scientists like Alexander von Humboldt and James Hutton made early contributions to understanding Earth as a dynamic system. Humboldt's work emphasized the interconnectedness of nature, while Hutton's geological theories laid the groundwork for recognizing Earth's long-term processes.

\subsection{Emergence of Climatology and Meteorology (19th Century)}

The study of weather and climate began to gain prominence in the 19th century. The establishment of meteorological observatories and the collection of data on atmospheric phenomena laid the foundation for understanding Earth's climate system.
System Thinking (20th Century):

The mid-20th century saw the development of system thinking, where scientists began to view Earth as a collection of interacting components rather than isolated processes. The advent of computers and modeling techniques facilitated a more holistic understanding of Earth's systems.

\subsection{Global Environmental Awareness (1970s)}

Increased awareness of environmental issues, such as pollution, deforestation, and climate change, led to the formation of the field of environmental science. Scientists recognized the need for an integrated approach to study Earth's complex interactions.
Development of Earth System Science (1980s-1990s):

In the 1980s and 1990s, Earth system science as a formal discipline began to take shape. The International Geosphere-Biosphere Programme (IGBP) and other global research initiatives played a crucial role in fostering collaboration among scientists from various fields.


\subsection{Key Reports and Frameworks (Late 20th Century)}

The 1992 Earth Summit in Rio de Janeiro marked a significant event, leading to the creation of the Intergovernmental Panel on Climate Change (IPCC) and the adoption of the Framework Convention on Climate Change. The IPCC's assessments and reports have since become instrumental in shaping our understanding of Earth's climate system.
Advancements in Technology (Late 20th Century to Present):

Advances in satellite technology, remote sensing, and computational modeling have provided scientists with the tools to observe and simulate Earth's complex systems more accurately.


\subsection{Ongoing Research and Challenges (21st Century)}

Earth system science continues to evolve, with ongoing research addressing contemporary challenges such as climate change, biodiversity loss, and sustainable development. Interdisciplinary collaborations are crucial in addressing the complex and interconnected issues facing our planet.
Earth system science remains a dynamic and evolving field as scientists strive to deepen their understanding of the Earth as a complex, integrated system.




\end{document}
